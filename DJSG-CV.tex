%------------------------------------
% Forked from Dario Taraborelli:
% http://nitens.org/taraborelli/cvtex
% DISCLAIMER: This template is provided for free and without any guarantee 
% that it will correctly compile on your system if you have a non-standard  
% configuration.
% Some rights reserved: http://creativecommons.org/licenses/by-sa/3.0/
%------------------------------------

%!TEX TS-program = xelatex
%!TEX encoding = UTF-8 Unicode

\documentclass[10pt, a4paper]{article}
\usepackage{fontspec} 

% DOCUMENT LAYOUT
\usepackage{geometry} 
\geometry{a4paper, textwidth=5.5in, textheight=8.5in, marginparsep=7pt, marginparwidth=.6in}
\setlength\parindent{0in}

% FONTS
\usepackage[usenames,dvipsnames]{xcolor}
\usepackage{xunicode}
\usepackage{xltxtra}
\defaultfontfeatures{Mapping=tex-text}
%\setromanfont [Ligatures={Common}, Numbers={OldStyle}, Variant=01]{Linux Libertine O}
%\setmonofont[Scale=0.8]{Monaco}
%%% modified by Karol Kozioł for ShareLaTeX use
\setmainfont[
  Ligatures={Common}, Numbers={OldStyle}, Variant=01,
  BoldFont=LinLibertine_RB.otf,
  ItalicFont=LinLibertine_RI.otf,
  BoldItalicFont=LinLibertine_RBI.otf
]{LinLibertine_R.otf}
\setmonofont[Scale=0.8]{DejaVuSansMono.ttf}

% ---- CUSTOM COMMANDS
\chardef\&="E050
\newcommand{\html}[1]{\href{#1}{\scriptsize\textsc{[html]}}}
\newcommand{\pdf}[1]{\href{#1}{\scriptsize\textsc{[pdf]}}}
\newcommand{\doi}[1]{\href{#1}{\scriptsize\textsc{[doi]}}}
% ---- MARGIN YEARS
\usepackage{marginnote}
\newcommand{\amper{}}{\chardef\amper="E0BD }
\newcommand{\years}[1]{\marginnote{\scriptsize #1}}
\renewcommand*{\raggedleftmarginnote}{}
\setlength{\marginparsep}{7pt}
\reversemarginpar

% HEADINGS
\usepackage{sectsty} 
\usepackage[normalem]{ulem} 
\sectionfont{\mdseries\upshape\Large}
\subsectionfont{\mdseries\scshape\normalsize} 
\subsubsectionfont{\mdseries\upshape\large} 

% PDF SETUP
% ---- FILL IN HERE THE DOC TITLE AND AUTHOR
\usepackage[%dvipdfm, 
bookmarks, colorlinks, breaklinks, 
% ---- FILL IN HERE THE TITLE AND AUTHOR
	pdftitle={Daniel J. Short Gianotti - vita},
	pdfauthor={Daniel J. Short Gianotti},
	pdfproducer={http://github.com/dgianotti/CV}
]{hyperref}  
\hypersetup{linkcolor=blue,citecolor=blue,filecolor=black,urlcolor=MidnightBlue} 

% DOCUMENT
\begin{document}
{\LARGE Daniel J. Short Gianotti}\\[1cm]
Department of Earth \& Environment\\
Boston University\\
685 Commonwealth Avenue\\
Boston, MA \texttt{02215}
U.S.A.\\[.2cm]
%Phone: \texttt{(206) 914-8269}\\
email: \href{mailto:gianotti@bu.edu}{gianotti@bu.edu}\\
\textsc{url}: \href{http://people.bu.edu/gianotti}{http://people.bu.edu/gianotti/}\\ 
%\vfill

%%\hrule
\section*{Current position}
\emph{PhD Student}, Boston University, Boston

%%\hrule
\section*{Areas of specialization}
Hydroclimatology $\cdot$ Ecohydrology $\cdot$ Preciptation Predictability \\
Stochastic Modeling $\cdot$ Statistical Representations of Variability

%\hrule
%\section*{About}
%Dan is a PhD student in Boston University's Department of Earth and Environment.  He studies water and climate, focusing right now on how well we can predict the future of water (particularly rain) in the context of our complicated climate.  He is particularly interested in how these predictions can help those in water-scarce areas manage the needs of human communities, agriculture, and ecosystems. He grew up in Alaska, has a B.S. in math from Harvey Mudd College, and has spent many years teaching everything from physics and creative writing to bicycle riding and banjo.

%\hrule
\section*{Education}
\noindent
\years{In Progress}\textsc{PhD} in Geography and Environment, Boston University\\
%\years{2003-2005}Coursework in Environmental Science and Engineering, University of Alaska, Anchorage\\
\years{2003}\textsc{BS} in Mathematics, Harvey Mudd College

%\hrule
%\section*{Grants, honors \& awards}
%\noindent
%\years{1921}Nobel Prize in Physics, Nobel Foundation

\section*{Publications \& talks}

\subsection*{Journal articles}
\noindent
\years{2015c}Anderson, BT, \textbf{DJ Short Gianotti}, \& GD Salvucci (2015), ``Detectability of historical trends in station-based precipitation characteristics over the continental United States,'' Journal of Geophysical Research (\emph{in review}).\\
\years{2015b}Gill, AL, AS Gallinat, R Sanders-DeMott, AJ Rigden, \textbf{DJ Short Gianotti}, JA Mantooth, \& PH Templer (2015), ``Changes in Autumn Senescence in Northern Hemisphere Deciduous Trees: a Meta-Analysis of Autumn Phenology Studies,'' Annals of Botany, Special Issue on Plants and Climate Change (\emph{in press}).\\
\years{2015a}Anderson, BT, \textbf{D Gianotti}, \& G Salvucci (2015), ``Characterizing the potential predictability of seasonal, station-based heavy precipitation accumulations and extreme dry-spell durations,'' Journal of Hydrometeorology 16 (2) 843-856.\\
\years{2014}\textbf{Short Gianotti, DJ}, BT Anderson, \& GD Salvucci (2014), ``The Potential Predictability of Precipitation Occurrence, Intensity, and Seasonal Totals over the Continental United States,'' Journal of Climate 27 (18), 6904-6918.\\
\years{2013b}Pal, I, BT Anderson, GD Salvucci, \& \textbf{DJ Gianotti} (2013), ``Shifting seasonality and increasing frequency of precipitation in wet and dry seasons across the US,'' Geophysical Research Letters 40 (15), 4030-4035.\\
\years{2013a}\textbf{Gianotti, D}, BT Anderson, \& GD Salvucci (2013), ``What Do Rain Gauges Tell Us about the Limits of Precipitation Predictability?'' Journal of Climate 26 (15), 5682-5688.

\subsection*{Conference presentations}
\noindent
\years{2015}\textbf{Short Gianotti, DJ}*, GD Salvucci, \& BT Anderson (2015) ``California Drought, Weather Variability, and Climate Variability,'' AGU Chapman Conference on California Drought: Causes, Impacts, and Policy, Irvine CA.\\
\years{2014}\textbf{Short Gianotti, DJ}*, BT Anderson, \& GD Salvucci (2014) ``Characterizing weather and climate variability for precipitation: A data-based stochastic modeling framework,'' American Geophysical Union Fall Meeting, San Francisco CA.\\
\years{2014}\textbf{Short Gianotti, DJ}*, BT Anderson, \& GD Salvucci (2014) ``Stochastic analysis of California's recent precipitation drought in the context of the last one hundred years,'' American Geophysical Union Fall Meeting, San Francisco CA.\\
\years{2014}Dietze, M*, HE Emery, D Gergel, \textbf{D Gianotti}, JA Mantooth, AJ Rigden (2014), ``Integrating satellite and tower phenology: a case-study in real-time ecological forecasting'' American Geophysical Union Fall Meeting, San Francisco CA.\\
\years{2014}Dietze, M*, HE Emery, D Gergel, \textbf{D Gianotti}, JA Mantooth, AJ Rigden (2014), ``Predicting phenology: A case-study in real-time ecological forecasting,'' Ecological Society of America Annual Meeting, Sacramento CA.\\
\years{2013}\textbf{Gianotti, DJ}*, BT Anderson, \& GD Salvucci  (2013), ``Potential Predictability of Precipitation: Occurrence or Intensity?'' 38th Climate Diagnostic and Prediction Workshop, College Park MD.\\
\years{2012}\textbf{Gianotti, DJ}*, BT Anderson, \& GD Salvucci (2012), ``Establishing Potential Predictability of U.S. Precipitation Using Rain Gauge Data,'' 37th Climate Diagnostic and Prediction Workshop, Fort Collins CO.\\
\years{2012}Pal, I*, BT Anderson, G Salvucci, \& \textbf{D Gianotti}  (2012), ``Magnitude and significance of observed trends in precipitation frequency over the U.S.,'' 37th Climate Diagnostic and Prediction Workshop, Fort Collins CO.\\
\years{2012}Anderson, BT*, \textbf{D Gianotti}, \& GD Salvucci (2012), ``Historical expansion of the summertime monsoon over the southwestern United States: What can regional models tell us about its causes?'' Regional Spectral Modeling Workshop, Scripps Institution of Oceanography, San Diego CA.\\
\years{2012}Pal, I*, BT Anderson, G Salvucci, \& \textbf{D Gianotti} (2012), ``Magnitude and significance of observed trends in precipitation frequency over the US,'' American Geophysical Union Fall Meeting, San Francisco CA.\\
\years{2011}\textbf{Gianotti, D}*, BT Anderson, \& G Salvucci (2011), ``Stochastic and deterministic aspects of observed seasonal-mean precipitation variations and extreme event occurrences over the United States,'' American Geophysical Union Fall Meeting, San Francisco CA.\\
\years{2011} Anderson, BT*, \textbf{D Gianotti}, \& GD Salvucci (2011), ``Detection of historical summertime monsoon precipitation variations and trends over the southwestern United States,'' WCRP Open Science Conference, Denver CO.\\
\years{2011} Anderson, BT*, D Gianotti, \& GD Salvucci (2011), ``Detection of historical precipitation variations and trends over the continental United States,'' Department of Energy Principal Investigators Meeting, Washington DC.\\
\years{2007}Schubert, DH*, \textbf{DJ Gianotti}, \& K Sauers (2007), ``Upgrades to a  wastewater lagoon treatment system in a rural sub-Arctic community in Alaska,'' International Symposium on Cold Region Development, Tampere Finland.\\
\years{2007}Schubert, DH*, \textbf{DJ Gianotti}, \& G Jones (2007), ``Application of a Thermal-hydraulic Model to Analyze and Design a Circulating Water System in Alaska,'' International Symposium on Cold Region Development, Tampere Finland.\\
\years{2005}\textbf{Gianotti, DJ}*, C Woolard, \& D White (2005),``Wastewater treatment lagoon design in rural Alaska,'' 45th Alaska Water and Wastewater Management Association Annual Statewide Conference, Juneau AK.\\
% Bruce AMS?
\\ \emph{* denotes presenting author}

\subsection*{Non-refereed research documents}
\noindent
%\years{}Engineering reports
\years{2007}Schubert, DH, \textbf{DJ Gianotti}, \& K Sauers (2007), ``Upgrades to a  wastewater lagoon treatment system in a rural sub-Arctic community in Alaska,'' Proceedings of the 8th International Symposium on Cold Region Development.\\
\years{2007}Schubert, DH, \textbf{DJ Gianotti}, \& G Jones (2007), ``Application of a Thermal-hydraulic Model to Analyze and Design a Circulating Water System in Alaska,'' Proceedings of the 8th International Symposium on Cold Region Development.\\
\years{2005}Woolard, C, \textbf{D Gianotti}, K Hardie, D White, \& A Pinto (2005), ``Waste Stabilization Pond Design and Performance Study,'' Prepared for the Alaska Department of Environmental Conservation.\\
\years{2003}\textbf{Gianotti, DJ} (2003), ``Fluid drop coalescence in a Hele-Shaw cell,'' Undergraduate Mathematics Thesis, Advised by A Nadim, \emph{Harvey Mudd College}.\\
\years{2002}Lampe, K, K Hultman, K Hedstrom, \textbf{D Gianotti}, E Deyo, \& R Seat (2002), ``Internal metrology for the Space Interferometry Mission,'' Undergraduate Physics Clinic Report, Advised by R Haskell, D MacDonald, \& B Nemati, \emph{Harvey Mudd College \& NASA-JPL}.

\subsection*{Non-conference presentations}
%\years{2015}Dan EE student presentations
\years{2014}\textbf{Gianotti, DJ}, (2014) ``Real weather, fake weather, and the California Drought,'' Dept. of Earth \& Env. Graduate Student Presentations, Boston University \\ %apr 11 2014
\years{2012}\textbf{Gianotti, DJ}, (2012) ``How predictable is rain?'' Dept. of  Geography \& Env. Graduate Student Presentations, Boston University \\
\years{2012} \textbf{Gianotti, D}, BT Anderson, \& G Salvucci (2012), ``Stochastic and deterministic aspects of observed seasonal-mean precipitation variations and extreme event occurrences over the United States,'' Science and Engineering Research Symposium, Boston University.


%%\hrule
\section*{Appointments held}
\noindent
\years{2011-2015}Research Assistant, Boston University\\
\years{2011}Math Teacher, Boston Public Schools\\
\years{2004-2010}Private Tutor, Anchorage \& Los Angeles\\
\years{2007-2008}Lab Technician, California Institute of Technology\\
\years{2005-2006}Environmental Engineering Associate, GV Jones \& Associates\\
\years{2004-2005}Research Assistant, University of Alaska, Anchorage\\
\years{2003-2005}Substitute Teacher, Anchorage School District\\
\years{2004}Staff, National Youth Science Camp\\
\years{2001-2003}Writing Consultant, Harvey Mudd College\\
\years{2002}Research Assistant, Lawrence Berkeley National Lab

%\hrule
%\section*{Teaching}
%\subsection*{Subjects taught}
%\begin{itemize}
%\item Mathematics through Multivariate Calculus and Linear Algebra
%\item Physics though Introductory Quantum Mechanics and Theoretical Mechanics
%\item Introductory Chemistry and Biology
%\item Writing including Creative, Technical, and Persuasive -- focuses on %Language and Structural Design
%\end{itemize}

%\hrule
\section*{Professional service}

\subsection*{Journal reviews}
Hydrology and Earth System Sciences

\subsection*{Memberships}
American Geophysical Union
%Boston Water Group

%\vspace{1cm}
\vfill{}
%\hrulefill

\begin{center}
{\scriptsize  Last updated: \today\- •\- 
% ---- PLEASE LEAVE THIS BACKLINK FOR ATTRIBUTION AS PER CC-LICENSE
Typeset in \href{http://nitens.org/taraborelli/cvtex}{
%\fontspec{Times New Roman}
\XeTeX }\\
% ---- FILL IN THE FULL URL TO YOUR CV HERE
\href{http://www.github.com/dgianotti/CV}{http://www.github.com/dgianotti/CV}}
\end{center}

\end{document}