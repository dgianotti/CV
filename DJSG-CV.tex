%------------------------------------
% Forked from Dario Taraborelli:
% http://nitens.org/taraborelli/cvtex
% DISCLAIMER: This template is provided for free and without any guarantee 
% that it will correctly compile on your system if you have a non-standard  
% configuration.
% Some rights reserved: http://creativecommons.org/licenses/by-sa/3.0/
%------------------------------------

%!TEX TS-program = xelatex
%!TEX encoding = UTF-8 Unicode

\documentclass[10pt, a4paper]{article}
\usepackage{fontspec} 
\usepackage{hyphenat}

% DOCUMENT LAYOUT
\usepackage{geometry} 
\geometry{a4paper, textwidth=5.5in, textheight=8.5in, marginparsep=7pt, marginparwidth=.6in}
\setlength\parindent{0in}

% FONTS
\usepackage[usenames,dvipsnames]{xcolor}
\usepackage{xunicode}
\usepackage{xltxtra}
\defaultfontfeatures{Mapping=tex-text}
%\setromanfont [Ligatures={Common}, Numbers={OldStyle}, Variant=01]{Linux Libertine O}
%\setmonofont[Scale=0.8]{Monaco}
%%% modified by Karol Kozioł for ShareLaTeX use
\setmainfont[
  Ligatures={Common}, Numbers={OldStyle}, Variant=01,
  BoldFont=LinLibertine_RB.otf,
  ItalicFont=LinLibertine_RI.otf,
  BoldItalicFont=LinLibertine_RBI.otf
]{LinLibertine_R.otf}
\setmonofont[Scale=0.8]{DejaVuSansMono.ttf}

% ---- CUSTOM COMMANDS
\chardef\&="E050
\newcommand{\html}[1]{\href{#1}{\scriptsize\textsc{[html]}}}
\newcommand{\pdf}[1]{\href{#1}{\scriptsize\textsc{[pdf]}}}
\newcommand{\doi}[1]{\href{#1}{\scriptsize\textsc{[doi]}}}
\renewcommand{\emph}[1]{\textit{#1}}
\newcommand{\lbr}{\vspace*{4pt}}

% ---- MARGIN YEARS
\usepackage{marginnote}
\newcommand{\amper{}}{\chardef\amper="E0BD }
\newcommand{\years}[1]{\marginnote{\scriptsize #1}}
\renewcommand*{\raggedleftmarginnote}{}
\setlength{\marginparsep}{7pt}
\reversemarginpar

% HEADINGS
\usepackage{sectsty} 
\usepackage[normalem]{ulem} 
\sectionfont{\mdseries\upshape\Large}
\subsectionfont{\mdseries\scshape\normalsize} 
\subsubsectionfont{\mdseries\upshape\large} 

% PDF SETUP
% ---- FILL IN HERE THE DOC TITLE AND AUTHOR
\usepackage[%dvipdfm, 
bookmarks, colorlinks, breaklinks, 
% ---- FILL IN HERE THE TITLE AND AUTHOR
	pdftitle={Daniel J. Short Gianotti - vita},
	pdfauthor={Daniel J. Short Gianotti},
	pdfproducer={http://github.com/dgianotti/CV}
]{hyperref}  
\hypersetup{linkcolor=blue,citecolor=blue,filecolor=black,urlcolor=MidnightBlue} 



% DOCUMENT
\begin{document}
{\LARGE Daniel J. Short Gianotti}\\[1cm]
Parsons Laboratory\\
Massachusetts Institute of Technology\\
15 Vassar St., Building 48\\
Cambridge, MA \texttt{02139}
U.S.A.\\[.2cm]
%Phone: \texttt{(206) 914-8269}\\
email: \href{mailto:gianotti@mit.edu}{gianotti@mit.edu}\\
%\textsc{url}: \href{http://people.bu.edu/gianotti}{http://people.bu.edu/gianotti/}\\ 
%\vfill

%%\hrule
\section*{Current position}
\years{2016-Present}\emph{Postdoctoral Associate}, Massachusetts Institute of Technology
%\emph{Research Programmer}, Massachusetts Institute of Technology
%\emph{PhD Student}, Boston University

%%\hrule
\section*{Areas of specialization}
Hydroclimate $\cdot$ Ecohydrology $\cdot$ Terrestrial Climate Feedbacks\\
Precipitation Predictability $\cdot$ Water-Carbon-Energy Cycle Coupling

%\hrule
%\section*{About}
%Dan is a PhD student in Boston University's Department of Earth and Environment.  He studies water and climate, focusing right now on how well we can predict the future of water (particularly rain) in the context of our complicated climate.  He is particularly interested in how these predictions can help those in water-scarce areas manage the needs of human communities, agriculture, and ecosystems. He grew up in Alaska, has a B.S. in math from Harvey Mudd College, and has spent many years teaching everything from physics and creative writing to bicycle riding and banjo.

%\hrule
\section*{Education}
\noindent
\years{2011-2016}\textsc{PhD} in Geography and Environment, Boston University\\
\textbf{Dissertation Title:} \emph{The Potential Predictability of Precipitation over the Continental United States}\\
\textbf{Defense Date:} August 9, 2016\\
\textbf{Committee:} Bruce T.\ Anderson (primary advisor), Guido D.\ Salvucci, Michael C.\ Dietze, Dara Entekhabi, \& Anthony C.\ Janetos (chair)\\

%\years{2003-2005}Coursework in Environmental Science and Engineering, University of Alaska, Anchorage\\
\years{1999-2003}\textsc{BS} in Mathematics, Harvey Mudd College

%\hrule
%\section*{Grants, honors \& awards}
%\noindent
%\years{1921}Nobel Prize in Physics, Nobel Foundation

\section*{Publications \& talks}

\subsection*{Journal articles}
\noindent
%\years{2018h}\textbf{Short Gianotti, DJ}, BT Anderson, \& GD Salvucci (2018), ``Weather noise correctly dominates climate signals in global climate model precipitation,'' (\emph{in preparation}). \lbr % Nature!

% \years{2017k}Rigden, AJ, \textbf{DJ Short Gianotti}, \& KA McColl (2017) ``The decoupling parameter in time and space,'' (\emph{in preparation}).\\

\years{2018i}Akbar, R, \textbf{DJ Short Gianotti}, GD Salvucci, \& D Entekhabi (2018) ``Mapped Hydroclimatology of Evapotranspiration and Drainage Runoff Using SMAP Brightness Temperature Observations and Precipitation Information,'' Water Resources Research (\emph{submitted}).\lbr 

\years{2018h}\textbf{Short Gianotti, DJ}, GD Salvucci, R Akbar, K McColl, \& D Entekhabi (2018) ``Representative soil column water content and residence times from SMAP and GPM,'' Water Resources Research (\emph{in preparation}). \lbr

\years{2018g}\textbf{Short Gianotti, DJ}, BT Anderson, \& GD Salvucci (2018), ``A kernel\hyp{}auto\hyp{}regressive weather generator for improved subseasonal\hyp{}to\hyp{}seasonal precipitation statistics,''  Water Resources Research (\emph{in preparation}). \lbr

\years{2018f}Feldman, AF, \textbf{DJ Short Gianotti}, AG Konings, KA McColl, R Akbar, GD Salvucci, \& D Entekhabi (2018), ``Pulse-response vegetation water uptake is persistent across biomes,'' Nature Plants (\emph{in press}). \lbr

\years{2018e}\textbf{Short Gianotti, DJ}, AJ Rigden, GD Salvucci, \& D Entekhabi (2018) ``Satellite and station observations demonstrate water availability's effect on continental-scale evaporative and photosynthetic land surface dynamics,'' Water Resources Research (\emph{in review}). \lbr

\years{2018d}Rigden, AJ, GD Salvucci, D Entekhabi, \& \textbf{DJ Short Gianotti} (2018) ``Partitioning evapotranspiration over the continental United States using weather station data,'' Geophysical Research Letters 45 (18), 9605--9613. \lbr

\years{2018c}Akbar, R, \textbf{DJ Short Gianotti}, KA McColl, E Haghighi, GD Salvucci, \& D Entekhabi (2018), ``Estimation of landscape soil water losses from satellite observations of soil moisture,'' Journal of Hydrometeorology 19 (5), 871--889. \lbr

\years{2018b}Akbar, R, \textbf{DJ Short Gianotti}, KA McColl, E Haghighi, GD Salvucci, \& D Entekhabi (2018) ``Hydrological storage length-scales represented by remote sensing estimates of soil moisture and precipitation,'' Water Resources Research 54 (3), 1476--1492. \lbr

\years{2018a}Haighighi, E, \textbf{DJ Short Gianotti}, R Akbar, GD Salvucci, \& D Entekhabi (2018) ``Soil and atmospheric controls on the land surface energy balance: A generalized framework for distinguishing moisture- and energy-limited evaporation regimes,'' Water Resources Research 53 (3), 1831--1851. \lbr

\years{2017b} McColl, K, W Wang, B Peng, R Akbar, \textbf{D Short Gianotti}, M Pan, \& D Entekhabi (2017), ``Global characterization of surface soil moisture drydowns,'' Geophysical Research Letters 44 (8), 3682--3690. \lbr

\years{2017a}Anderson, BT, JC Furtado, E Di Lorenzo, \textbf{DJ Short Gianotti} (2017), ``Tracking the Pacific Decadal Precession,'' Journal of Geophysical Research: Atmospheres 122 (6) 3214--3227. \lbr

\years{2016b} Anderson, BT, \textbf{DJ Short Gianotti}, GD Salvucci, \& J Furtado (2016), ``Dominant timescales of potentially predictable precipitation variations across the continental United States,'' Journal of Climate 29, 8881--8897. \lbr

\years{2016a}Anderson, BT, \textbf{DJ Short Gianotti}, J Furtado, \& E Di Lorenzo (2016), ``A decadal precession of atmospheric pressures over the North Pacific,'' Geophysical Research Letters 43 (8) 3921--3927. \lbr

\years{2015c}Anderson, BT, \textbf{DJ Short Gianotti}, \& GD Salvucci (2015), ``Detectability of historical trends in station-based precipitation characteristics over the continental United States,'' Journal of Geophysical Research 120 (10) 4842--4859. \lbr

\years{2015b}Gill, AL, AS Gallinat, R Sanders-DeMott, AJ Rigden, \textbf{DJ Short Gianotti}, JA Mantooth, \& PH Templer (2015), ``Changes in Autumn Senescence in Northern Hemisphere Deciduous Trees: a Meta-Analysis of Autumn Phenology Studies,'' Annals of Botany, (Special Issue on Plants and Climate Change) 116, 875--888. \lbr

\years{2015a}Anderson, BT, \textbf{D Gianotti}, \& G Salvucci (2015), ``Characterizing the potential predictability of seasonal, station-based heavy precipitation accumulations and extreme dry-spell durations,'' Journal of Hydrometeorology 16 (2) 843--856. \lbr

\years{2014}\textbf{Short Gianotti, DJ}, BT Anderson, \& GD Salvucci (2014), ``The Potential Predictability of Precipitation Occurrence, Intensity, and Seasonal Totals over the Continental United States,'' Journal of Climate 27 (18), 6904--6918. \lbr

\years{2013b}Pal, I, BT Anderson, GD Salvucci, \& \textbf{DJ Gianotti} (2013), ``Shifting seasonality and increasing frequency of precipitation in wet and dry seasons across the US,'' Geophysical Research Letters 40 (15), 4030--4035. \lbr

\years{2013a}\textbf{Gianotti, D}, BT Anderson, \& GD Salvucci (2013), ``What Do Rain Gauges Tell Us about the Limits of Precipitation Predictability?'' Journal of Climate 26 (15), 5682--5688. \lbr

\subsection*{Conference presentations}
\noindent
\years{2018}\textbf{Short Gianotti, DJ}, GD Salvucci, AJ Rigden, D Entekhabi* (2018) ``Linkages between water, energy and carbon cycles revealed by SMAP,'' SMAP End of Prime Mission Science Meeting, Jet Propulsion Laboratory, Pasadena, CA. \lbr % April 18, 2018

\years{2018}Akbar, R, \textbf{Short Gianotti, DJ}*, K McColl, E Haghighi, GD Salvucci, D Entekhabi (2018) ``Estimation of ecosystem-scale soil water losses from satellite observations of soil moisture,'' SMAP End of Prime Mission Science Meeting, Jet Propulsion Laboratory, Pasadena, CA. \lbr % April 18, 2018

\years{2017}\textbf{Short Gianotti, DJ}*, AJ Rigden, GD Salvucci, \& D Entekhabi (2017) ``Effects of water availability through the coupled land-atmosphere system,'' American Geophysical Union Fall Meeting: H12G-07, New Orleans, LA.\lbr

\years{2017} Haghighi, E*, \textbf{Short Gianotti, DJ}, R Akbar, GD Salvucci, \& D Entekhabi (2017) ``What determines transitions between energy- and moisture-limited evaporative regimes?'' American Geophysical Union Fall Meeting: H44C-07, New Orleans, LA.\lbr

\years{2017}Salvucci, GD*, AJ Rigden, \textbf{DJ Short Gianotti}, \& D Entekhabi (2017) ``Soil moisture (SMAP) and vapor pressure deficit controls on evaporation fraction over the Continental U.S.,'' American Geophysical Union Fall Meeting: H12G-01, New Orleans, LA.\lbr

\years{2017}\textbf{Short Gianotti, DJ}*, AJ Rigden, GD Salvucci, \& D Entekhabi (2017) ``Soil moisture controls on water/energy/carbon coupling,'' Science Utilization of SMAP Meeting, Cambridge, MA.\lbr

\years{2017}Akbar, R*, \textbf{DJ Short Gianotti}, E Haighighi, GD Salvucci, \& D Entekhabi (2017) ``Estimation of ecosystem-scale soil water losses from satellite observations of soil moisture,'' Science Utilization of SMAP Meeting, Cambridge, MA.\lbr % October 19, 2017

\years{2017}Entekhabi, D*, SMAP Science Team, \textbf{DJ Short Gianotti}, Akbar, R, AJ Rigden, GD Salvucci, \& JS Kimball (2017) ``The Science Applications of SMAP,'' Science Utilization of SMAP Meeting, Cambridge, MA.\lbr % October 19, 2017

\years{2016}\textbf{Short Gianotti, DJ}*, AJ Rigden, GD Salvucci, \& D Entekhabi (2016) ``Soil Moisture Controls on Evaporative Fraction,'' American Geophysical Union Fall Meeting: H24C-03, San Francisco, CA.\lbr

%\years{2015}\textbf{Short Gianotti, DJ}*, GD Salvucci, \& BT Anderson (2015) ``Potential Predictability of Precipitation from Rain Gauges to GCMs,'' Scales and Scaling in the Climate System: Bridging Theory, Climate Models, and Data, Montr\'{e}al Canada.\\
\years{2015}\textbf{Short Gianotti, DJ}*, GD Salvucci, \& BT Anderson (2015) ``California Drought, Weather Variability, and Climate Variability,'' AGU Chapman Conference on California Drought: Causes, Impacts, and Policy, Irvine CA.\lbr

\years{2014}\textbf{Short Gianotti, DJ}*, BT Anderson, \& GD Salvucci (2014) ``Characterizing weather and climate variability for precipitation: A data-based stochastic modeling framework,'' American Geophysical Union Fall Meeting, San Francisco CA.\lbr

\years{2014}\textbf{Short Gianotti, DJ}*, BT Anderson, \& GD Salvucci (2014) ``Stochastic analysis of California's recent precipitation drought in the context of the last one hundred years,'' American Geophysical Union Fall Meeting, San Francisco CA.\lbr

\years{2014}Dietze, M*, HE Emery, D Gergel, \textbf{D Gianotti}, JA Mantooth, \& AJ Rigden (2014), ``Integrating satellite and tower phenology: a case-study in real-time ecological forecasting'' American Geophysical Union Fall Meeting, San Francisco CA.\lbr

\years{2014}Dietze, M*, HE Emery, D Gergel, \textbf{D Gianotti}, JA Mantooth, \& AJ Rigden (2014), ``Predicting phenology: A case-study in real-time ecological forecasting,'' Ecological Society of America Annual Meeting, Sacramento CA.\lbr

\years{2013}\textbf{Gianotti, DJ}*, BT Anderson, \& GD Salvucci  (2013), ``Potential Predictability of Precipitation: Occurrence or Intensity?'' 38th Climate Diagnostic and Prediction Workshop, College Park MD.\lbr

\years{2012}\textbf{Gianotti, DJ}*, BT Anderson, \& GD Salvucci (2012), ``Establishing Potential Predictability of U.S. Precipitation Using Rain Gauge Data,'' 37th Climate Diagnostic and Prediction Workshop, Fort Collins CO.\lbr

\years{2012}Pal, I*, BT Anderson, G Salvucci, \& \textbf{D Gianotti}  (2012), ``Magnitude and significance of observed trends in precipitation frequency over the U.S.,'' 37th Climate Diagnostic and Prediction Workshop, Fort Collins CO.\lbr

\years{2012}Anderson, BT*, \textbf{D Gianotti}, \& GD Salvucci (2012), ``Historical expansion of the summertime monsoon over the southwestern United States: What can regional models tell us about its causes?'' Regional Spectral Modeling Workshop, Scripps Institution of Oceanography, San Diego CA.\lbr

\years{2012}Pal, I*, BT Anderson, G Salvucci, \& \textbf{D Gianotti} (2012), ``Magnitude and significance of observed trends in precipitation frequency over the US,'' American Geophysical Union Fall Meeting, San Francisco CA.\lbr

\years{2011}\textbf{Gianotti, D}*, BT Anderson, \& G Salvucci (2011), ``Stochastic and deterministic aspects of observed seasonal-mean precipitation variations and extreme event occurrences over the United States,'' American Geophysical Union Fall Meeting, San Francisco CA.\lbr

\years{2011} Anderson, BT*, \textbf{D Gianotti}, \& GD Salvucci (2011), ``Detection of historical summertime monsoon precipitation variations and trends over the southwestern United States,'' WCRP Open Science Conference, Denver CO.\lbr

\years{2011} Anderson, BT*, D Gianotti, \& GD Salvucci (2011), ``Detection of historical precipitation variations and trends over the continental United States,'' Department of Energy Principal Investigators Meeting, Washington DC.\lbr

\years{2007}Schubert, DH*, \textbf{DJ Gianotti}, \& K Sauers (2007), ``Upgrades to a  wastewater lagoon treatment system in a rural sub-Arctic community in Alaska,'' International Symposium on Cold Region Development, Tampere Finland.\lbr

\years{2007}Schubert, DH*, \textbf{DJ Gianotti}, \& G Jones (2007), ``Application of a Thermal-hydraulic Model to Analyze and Design a Circulating Water System in Alaska,'' International Symposium on Cold Region Development, Tampere Finland.\lbr

\years{2005}\textbf{Gianotti, DJ}*, C Woolard, \& D White (2005),``Wastewater treatment lagoon design in rural Alaska,'' 45th Alaska Water and Wastewater Management Association Annual Statewide Conference, Juneau AK.\lbr
% Bruce AMS?
\\ \emph{* denotes presenting author}

\subsection*{Non-refereed research documents}
\noindent
%\years{}Engineering reports
\years{2007}Schubert, DH, \textbf{DJ Gianotti}, \& K Sauers (2007), ``Upgrades to a  wastewater lagoon treatment system in a rural sub-Arctic community in Alaska,'' Proceedings of the 8th International Symposium on Cold Region Development.\lbr

\years{2007}Schubert, DH, \textbf{DJ Gianotti}, \& G Jones (2007), ``Application of a Thermal-hydraulic Model to Analyze and Design a Circulating Water System in Alaska,'' Proceedings of the 8th International Symposium on Cold Region Development.\lbr

\years{2005}Woolard, C, \textbf{D Gianotti}, K Hardie, D White, \& A Pinto (2005), ``Waste Stabilization Pond Design and Performance Study,'' Prepared for the Alaska Department of Environmental Conservation.\lbr

\years{2003}\textbf{Gianotti, DJ} (2003), ``Fluid drop coalescence in a Hele-Shaw cell,'' Undergraduate Mathematics Thesis, Advised by A Nadim, \emph{Harvey Mudd College}.\lbr

\years{2002}Lampe, K, K Hultman, K Hedstrom, \textbf{D Gianotti}, E Deyo, \& R Seat (2002), ``Internal metrology for the Space Interferometry Mission,'' Undergraduate Physics Clinic Report, Advised by R Haskell, D MacDonald, \& B Nemati, \emph{Harvey Mudd College \& NASA-JPL}.

%\subsection*{Non-conference presentations}
%\years{2016} \textbf{Short Gianotti, DJ}, (2016) ``The Potential Predictability of Precipitation over the Continental United States,'' Dissertation Defense, Boston University.\\ %Aug 9, 2016
%\years{2015} \textbf{Gianotti, DJ}, (2015) ``Weather models for climate variability,'' Dept. of Earth \& Env. Graduate Student Presentations, Boston University. \\ %apr 10 2015
%\years{2014}\textbf{Gianotti, DJ}, (2014) ``Real weather, fake weather, and the California Drought,'' Dept. of Earth \& Env. Graduate Student Presentations, Boston University. \\ %apr 11 2014
%\years{2012}\textbf{Gianotti, DJ}, (2012) ``How predictable is rain?'' Dept. of  Geography \& Env. Graduate Student Presentations, Boston University. \\
%\years{2012} \textbf{Gianotti, D}, BT Anderson, \& G Salvucci (2012), ``Stochastic and deterministic aspects of observed seasonal-mean precipitation variations and extreme event occurrences over the United States,'' Science and Engineering Research Symposium, Boston University.
%
\section*{Published software packages}
\noindent
\years{2016} \textbf{Short Gianotti, DJ} (2016) ``Occurrence Markov Chain daily precipitation model,'' \\
http://github.com/dgianotti/OMC-precip, DOI:10.5281/zenodo.45435.

%%\hrule
\section*{Appointments held}
\noindent
\years{2016-Present}Postdoctoral Associate, Massachusetts Institute of Technology\\
\years{2011-2015}Research Assistant, Boston University\\
\years{2011}Math Teacher, Boston Public Schools\\
\years{2004-2010}Private Tutor, Anchorage \& Los Angeles\\
\years{2007-2008}Lab Technician, California Institute of Technology\\
\years{2005-2006}Environmental Engineering Associate, GV Jones \& Associates\\
\years{2004-2005}Research Assistant, University of Alaska, Anchorage\\
\years{2003-2005}Substitute Teacher, Anchorage School District\\
\years{2004}Staff, National Youth Science Camp\\
\years{2001-2003}Writing Consultant, Harvey Mudd College\\
\years{2002}Research Assistant, Lawrence Berkeley National Lab

%\hrule
\section*{Teaching}
\textbf{Teaching Fellow:}\\
\hspace*{3em}\emph{Introduction to Quantitative Environmental Modeling} (Boston University)\\
\textbf{Guest Lecturer:}\\
\hspace*{3em}\emph{Introduction to Hydrology and Water Resources} (MIT)\\
\hspace*{3em}\emph{Introduction to Hydrologic Modeling} (MIT)\\
\textbf{K-12 Instruction:}\\
\hspace*{3em}High school mathematics (Boston Public Schools)\\
\hspace*{3em}All subjects, all ages (Substitute Teacher -- Anchorage School District)\\
\textbf{Private Tutoring:}\\
\hspace*{3em}All subjects through advanced secondary\\
\hspace*{3em}Chemistry, biology through introductory undergraduate\\
\hspace*{3em}Math, physics, writing through advanced undergraduate\\
%\subsection*{Subjects taught}
%\begin{itemize}
%\item Mathematics through Multivariate Calculus and Linear Algebra
%\item Physics though Introductory Quantum Mechanics and Theoretical Mechanics
%\item Introductory Chemistry and Biology
%\item Writing including Creative, Technical, and Persuasive -- focuses on %Language and Structural Design
%\end{itemize}

%\hrule
\section*{Professional development}
\years{2015} \emph{ComSciCon 2015} Communicating Science Workshop, Harvard University.

%\hrule
\section*{Professional service}

\subsection*{Journal reviews}
Bulletin of the American Meteorological Society\\
Hydrology and Earth System Sciences\\
Journal of Hydrometeorology\\
International Journal of Climatology

\subsection*{Memberships}
American Geophysical Union\\ % 2011-present
Boston Water Group % 2014-present
%Boston Area Hydrology Journal Club % 2012-present

%\vspace{1cm}
\vfill{}
%\hrulefill

\begin{center}
{\scriptsize  Last updated: \today\- •\- 
% ---- PLEASE LEAVE THIS BACKLINK FOR ATTRIBUTION AS PER CC-LICENSE
Typeset in \href{http://nitens.org/taraborelli/cvtex}{
%\fontspec{Times New Roman}
\XeTeX }\\
% ---- FILL IN THE FULL URL TO YOUR CV HERE
Latest CV version available at: \href{http://www.github.com/dgianotti/CV}{http://www.github.com/dgianotti/CV}}
\end{center}

\end{document}